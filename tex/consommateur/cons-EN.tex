% !TEX encoding = utf-8
\documentclass[handout]{beamer}
\usepackage[french]{babel}
\usepackage[T1]{fontenc}
\usepackage[utf8]{inputenc}
\usepackage{graphicx}
\usepackage{epstopdf}
% functions to plot
\def\func(#1){(#1)*(1-(#1))}
\hypersetup{colorlinks = true,linkcolor = blue,urlcolor  = blue}

\newenvironment{iPar}[1]{\textbf{#1} \begin{itemize}}{\end{itemize}}

\addtobeamertemplate{navigation symbols}{}{%
    \usebeamerfont{footline}%
    \usebeamercolor[fg]{footline}%
    \hspace{1em}%
    \insertframenumber/\inserttotalframenumber
}

\title{Consumer Choice}
\author{Microeconomics \\ 20851}
\date{}

\begin{document}

\frame{\titlepage}

\section[Outline]{}
\frame{\tableofcontents}

\section{Preferences}

\begin{frame}\frametitle{Preferences}
\textbf{Defined preferences in baskets of goods and services}
\begin{itemize} \item For two baskets defined as $A$ and $B$, preferences dictate the preferred one
\item Preferences are like hierarchical wish lists: ignore prices and resources for now, they will come later. 
\item Preference relations are denoted by $\succ,\succeq,\sim$. 
\end{itemize}

\end{frame}

\begin{frame}\frametitle{Important Axioms Concerning Preferences}

\textbf{Exhaustive: For any baskets A and B, te consumer either}
\begin{itemize}
\item  \textbf{always} prefers A over B ($A\succ B$)
\item  \textbf{always} prefers B over A ($B\succ A$)
\item is indifferent between A and B ($A \sim B$) \end{itemize}

\textbf{Is it restrictive?}
\begin{itemize}
\item Yes, e.g.: ice cream vs. soup
\item We can include circumstances within baskets
\item We then have the baskets (ice cream, heat), (ice cream, cold), (soup, heat), (soup, cold).
\end{itemize}
\end{frame}

\begin{frame}\frametitle{Important Axioms Concerning Preferences (II)}
\textbf{Transitivity}
\begin{itemize}
\item Given three baskets A, B, C  such as  $A\succ B$, $B \succ C$, 
then the consumer prefers A over C ($A \succ C$).
\end{itemize}

\textbf{Nonsatiation}\begin{itemize}
\item If A contains exactly as many goods as B plus one good, then $A \succ B$.
\item The weaker version ($A \succeq B$,indifference or preference) is reasonable  
\item Labelling goods: goods must always be desirable (air quality vs. pollution). 
\end{itemize}


\end{frame}

\begin{frame}\frametitle{Indifference Curves}
\textbf{Two goods $X,Y$}:
\begin{itemize}
\item For any given basket $(X_0,Y_0)$, combinations $(X,Y)$ such that the consumer is indifferent between $(X,Y)$ and $(X_0,Y_0)$
\item Indifference curves going towards the north-east indicate higher utility levels (nonsatiation)
\end{itemize}

\end{frame}

\begin{frame}\frametitle{Indifference Curves Don't Cross}

\textbf{Exercise A}: Show that indifference curves that don't intersect each other respect the transitivity hypothesis.
\end{frame}


\begin{frame}\frametitle{Marginal Rate of Substitution (MRS)}
\textbf{Definition}
\begin{itemize}
\item Goods:  $X$ (Food) and $Y$ (Heating)
\item For a given basket $(X_0, Y_0)$, MRS of $X$ as a function of $Y$: Heating ($Y$) that the consumer is willing to sacrifice in order to obtain a unit of food ($X$)
\item It's the value that a consumer gives to $X$ in terms of units of another good $Y$.
\item This corresponds to the slope of the indifference curve at the point $(X_0,Y_0)$.
\end{itemize}
\end{frame}

\begin{frame}\frametitle{Convexity of Indifference Curves}
\begin{itemize}
\item If the amount of food ($X$) increases, how does the MRS of $X$ in terms of $Y$ change?
\end{itemize}

This corresponds to the convexity of indifference curves. 
\end{frame}

\begin{frame}\frametitle{Utility}

\textbf{Representing preferences}
\begin{itemize}
\item  Utility functions: assign a number to each basket
\item $U$ represents preferences if and only if $ A \succ B \Rightarrow U(A) > U(B)$ and 
$ U(A) > U(B)   \Rightarrow A \succ B$
\end{itemize}
\vfill
\end{frame}

\begin{frame}{Utility}
\textbf{Preferences are ordinal (hierarchical)}
\begin{itemize}
\item If $f$ is a strictly increasing function and $U$ represents preferences, then $V(X) = f(U(X))$ represente the same preferences.

$$ U(X) > U(Y) \iff f(U(X)) > f(U(Y))$$
\item A value of utility has no meaning, the order of the baskets is what matters.
\item $U(X,Y) = \log X + \log Y$ and $V(X,Y) = XY$ represent the same preferences
\end{itemize}
\textbf{Exercise B}: Show that $U$ and $V$ have the same preferences. 
\end{frame}


\begin{frame}{Back to the MRS}
\textbf{Context}
\begin{itemize}
\item Two goods, $X$, $Y$.  Preferences are represented by the utility function $U(X,Y)$
\item e.g. $U(X,Y) = \log X + \log Y$
\end{itemize}
\textbf{MRS of $X$ in terms of $Y$} \pause
\begin{itemize}
\item How much $Y$ is sacrificed for more $X$\pause
\item Formally: if $X$ increases by $\Delta X$: what is the change $\Delta Y $ that will keep the indifference?
\end{itemize}
\end{frame}

\begin{frame}{Calculating the MRS}
\textbf{MRS of $X$ in terms of $Y$} \pause
\begin{itemize}
\item We need a $\Delta Y$ such that $U(X + \Delta X, Y + \Delta Y) = U(X,Y)$
\item First order approximation: $$U(X+\Delta X, Y+ \Delta Y) \simeq  U(X,Y)+  \Delta X  \frac{\partial U}{\partial X} + \Delta Y \frac{\partial U}{\partial Y}$$\pause
$$ \Rightarrow \;\; MRS = \Delta Y/ \Delta X =  -\frac{\partial U}{\partial X}/ \frac{\partial U}{\partial Y}$$
\end{itemize}

\textbf{Example}
\begin{itemize}
\item $U(X,Y) = \log X + \log Y$
 $$MRS = \frac{\partial U}{\partial X}/ \frac{\partial U}{\partial Y} = -Y/X$$
\end{itemize}

\end{frame}

\begin{frame}{By Total Differential}

Take the total differential:
\begin{align}
dU = \frac{\partial U}{\partial X}dX + \frac{\partial U}{\partial Y}dY
\end{align}
Set $dU = 0$, then 

\begin{align}
\frac{dY}{dX}\bigg\rvert_{dU=0} = -\frac{\partial U}{\partial X}/ \frac{\partial U}{\partial Y}
\end{align}

\end{frame}


\section{Budget constraint}

\begin{frame}{Budget Constraint}
\begin{itemize}
\item We can't spend more than what we own $I$
\item Two goods $X$, $Y$:  Constraints: $p_X X + p_Y Y = I$ \\ Defines what is affordable given $I$\\
\item Solve for $Y$ in terms of  $X$:     $Y = \frac{I - p_X X}{p_Y}$\\
The exchange rate between $X$ and $Y$ while respecting the constraint: $$\frac{dY}{dX} = -\frac{p_X}{p_Y}$$
\end{itemize}

\end{frame}

\begin{frame}\frametitle{Constraints}
\textbf{Normalization}
\begin{itemize}
\item  The budget constraint is the same if we simultaneously multiply the prices and the budget by $\lambda$.
\item We can buy the same goods.\pause
\item Normalizing $p_Y = 1$. Then $Y = I - p_X X$. $p_X$ is now in terms of quantity of $Y$ (called the \textit{numeraire}). 
\end{itemize}
\textbf{Exercise C}: Show that a budget constraint doesn't change if we multiply prices and income by $\lambda>0$. 
\end{frame}

\section{Consumer choices}

\begin{frame}\frametitle{Consumer Choices}

\begin{itemize}
\item The constraint is a fixed value. What is the highest utility level on this constraint?

\item We cannot be on a utility curve that is higher than the constraint

\item Any lower indifference curve is sub-optimal.

\item The curve touching the constraint (often tangent) wil give us the highest level of well-being achievable
\end{itemize}

\end{frame}

\section{Mathematical conditions}

\begin{frame}\frametitle{The Direct Approach}

\textbf{The problem}
\begin{itemize}
\item Maximize $U(X,Y)$ given the following constraint: $p_X X+ p_YY = I$
\end{itemize}

\pause
\textbf{Step 1: Substitute the constraint}
\begin{itemize}
\item If we buy $X$ then we consume $Y(X) = \frac{I - p_X X}{p_Y}$
\item Utility is only a function of $X$: $V(X) = U(X,Y(X))$
\end{itemize}

\pause

\textbf{Step 2: Maximize without a constraint}
\begin{itemize}
\item Check that the corner solution is not optimal (say $X= 0$ and $Y=0$)
\item Take the first order condition 
\end{itemize}
\end{frame}

\begin{frame}{The Direct Approach}

\textbf{The FOC}
\begin{itemize}
\item $$\frac{dV}{dX} = 0 \iff \frac{dU}{dX} + \frac{dY}{dX}\frac{dU}{dY} = 0$$
 $$\iff \frac{dU}{dX}\Bigg/\frac{dU}{dY} = \frac{p_X}{p_Y}$$
\end{itemize}
MRS on the indifference curve = Slope of the budget constraint
\end{frame}

\begin{frame}\frametitle{Solution}

\textbf{Exercise D}: Find the demands for $u(X,Y) = XY$ subject to $p_X X + p_Y Y \le I $. 
\end{frame}

\section{General approach}

\begin{frame}{Lagrangian I}
We can set the lagrangian

\begin{align*}
L(X,Y,\lambda) = U(X,Y) - \lambda (p_X X + p_Y Y - I)
\end{align*}

\pause

If we maximize: $\max_{X,Y,\lambda} L(X,Y,\lambda)$, the FOC are

\pause

\begin{align*}
U'_X(X,Y) - \lambda p_X = 0 \\
U'_Y(X,Y) - \lambda p_Y = 0 \\
p_X X + p_Y Y = I
\end{align*}
\end{frame}

\begin{frame}{Lagrangian II}
Taking the ratio of the first two FOC, we get:

\begin{align*}
\frac{U'_X(X,Y)}{U'_Y(X,Y)} = \frac{p_X}{p_Y} \\
p_X X + p_Y Y = I
\end{align*}

\textbf{Exercise E}: Find the demands for $u(X,Y) = XY$ this time using a lagrangian. 
\newline

\end{frame}

\begin{frame}{Indirect utility}

The indirect utility $V(p_X,p_Y,I)$ is the highest utility level achievable with the prices $(p_X,p_Y)$ and 
$$ V(p_X,p_Y,I) = \max_{X,Y} \{ u(X,Y) : p_X X + p_Y Y \le I\} $$

\textbf{Exercise F}: Show that $\frac{\partial V}{\partial I} = \lambda$ where $V$ is the indirect utility. 
\end{frame}

\begin{frame}{Roy's Identity}

If the indirect utility is given by $V(p_X,p_Y,I)$ then we can find the demands with Roy's Identity: 

$$ X^*(p_X,p_Y,I) = \frac{\partial{-V(p_X,p_Y,I)}/\partial{p_X}}{\partial{V(p_X,p_Y,I)}/\partial{I}}  $$

\textbf{Exercise F}: Show that this is true using the envelope theorem. 
\end{frame}

\end{document}
