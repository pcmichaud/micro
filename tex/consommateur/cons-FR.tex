% !TEX encoding = utf-8
\documentclass[handout]{beamer}
\usepackage[french]{babel}
\usepackage[T1]{fontenc}
\usepackage[utf8]{inputenc}
\usepackage{graphicx}
\usepackage{epstopdf}
% functions to plot
\def\func(#1){(#1)*(1-(#1))}
\hypersetup{colorlinks = true,linkcolor = blue,urlcolor  = blue}

\newenvironment{iPar}[1]{\textbf{#1} \begin{itemize}}{\end{itemize}}

\addtobeamertemplate{navigation symbols}{}{%
    \usebeamerfont{footline}%
    \usebeamercolor[fg]{footline}%
    \hspace{1em}%
    \insertframenumber/\inserttotalframenumber
}

\title{Choix du consommateur}
\author{Microéconomie \\ 20851}
\date{}

\begin{document}

\frame{\titlepage}

\section[Outline]{}
\frame{\tableofcontents}

\section{Les préférences}

\begin{frame}\frametitle{Préférences}
\textbf{Préférences définies sur des paniers de biens et services}
\begin{itemize} \item Panier: vecteur de quantités $x = (x_1, x_2,\cdots,x_n)$.
\item Pour deux paniers $A$ and $B$, préférences dictent lequel est préféré
\item Préférences sont comme liste de souhait (hierarchisée): ignore les prix et les ressources. Viendra plus tard. 
\item Les relations de préférences sont dénotées par $\succ,\succeq,\sim$. 
\end{itemize}

\end{frame}

\begin{frame}\frametitle{Hypothèses importantes sur les préférences}

\textbf{Exhaustive: Pour tous paniers A et B soit le consommateur}
\begin{itemize}
\item  préfère \textbf{toujours} A à B ($A\succ B$)
\item  préfère \textbf{toujours} B à A ($B\succ A$)
\item est indifférent entre A et B ($A \sim B$) \end{itemize}

\textbf{Est-ce restrictif?}
\begin{itemize}
\item Oui, e.g.: crème glacée vs. soupe
\item On peut inclure les circonstances dans les paniers
\item Les paniers sont alors (crème glacée, chaleur), (crème glacée, froid), (soupe, chaleur), (soupe, froid).
\end{itemize}
\end{frame}

\begin{frame}\frametitle{Hypothèses importantes sur les préférences (II)}
\textbf{Transitivité}
\begin{itemize}
\item Si trois paniers A, B, C  tels que  $A\succ B$, $B \succ C$, 
alors le consommateur préfère A à C ($A \succ C$).
\end{itemize}

\textbf{Non-satiation}\begin{itemize}
\item Si A contient au moins autant que le panier B, strictement plus d'au moins un bien dans le panier,
alors $A \succ B$.
\item Version faible ($A \succeq B$,indifférence ou préférence) est raisonnable  
\item Étiquette des biens: les biens sont désirables (qualité air vs. pollution). 
\end{itemize}


\end{frame}

\begin{frame}\frametitle{Courbes d'indifférence}
\textbf{Deux biens $X,Y$}:
\begin{itemize}
\item Pour tout panier $(X_0,Y_0)$, combinaisons $(X,Y)$ tel que consommateur indifférent entre $(X,Y)$ and $(X_0,Y_0)$
\item Courbe indifférence vers nord-est indique niveau utilité plus élevé (non-satiation)
\end{itemize}

\end{frame}

\begin{frame}\frametitle{Les courbes d'indifférence ne se croisent pas}

\textbf{Exercice A}: Montrer que des courbes d'indifférence qui se croisent ne respectent pas la transitivité.
\end{frame}


\begin{frame}\frametitle{Taux marginal de substitution (TMS)}
\textbf{Definition}
\begin{itemize}
\item Biens:  $X$ et $Y$
\item Pour un panier $(X_0, Y_0)$, TMS de $X$ en fonction de $Y$: Quantité de chauffage ($Y$) que le consommateur est prêt à sacrifier pour avoir une unité de plus de nourriture ($X$)
\item Valeur que le consommateur porte sur un bien $X$ en termes d'unité d'un autre bien $Y$.
\item Correspond à la pente de la courbe d'indifférence à $(X_0,Y_0)$.
\end{itemize}
\end{frame}

\begin{frame}\frametitle{Convexité des courbes d'indifférences}
\begin{itemize}
\item Si la quantité de nourriture ($X$) augmente comment le TMS de $X$ en fonction de $Y$ change?
\end{itemize}

Correspond à la convexité des courbes d'indifférences. 
\end{frame}

\begin{frame}\frametitle{Utilité}

\textbf{Représentation des préférences}
\begin{itemize}
\item  Fonction d'utilité: assigne un nombre à chaque panier
\item $U$ représente les préférences si et seulement si $ A \succ B \Rightarrow U(A) > U(B)$ et 
$ U(A) > U(B)   \Rightarrow A \succ B$
\end{itemize}
\vfill
\end{frame}

\begin{frame}{Utilité}
\textbf{Les préférences sont ordinales (hiérarchiques)}
\begin{itemize}
\item Si $f$ est une fonction strictement croissance et $U$ représente des préférences, alors $V(X) = f(U(X))$ représente les même préférences.

$$ U(X) > U(Y) \iff f(U(X)) > f(U(Y))$$
\item La valeur de l'utilité n'a pas de signification, l'ordonnancement des paniers est important.
\item $U(X,Y) = \ln X + \ln Y$ et $V(X,Y) = XY$ représente les mêmes préférences
\end{itemize}
\textbf{Exercice B}: Montrer que $U$ et $V$ ont les mêmes préférences. 
\end{frame}


\begin{frame}{Retour vers le TMS}
\textbf{Contexte}
\begin{itemize}
\item Deux biens, $X$, $Y$.  Préférences représentées par la fonction d'utilité $U(X,Y)$
\item e.g. $U(X,Y) = \ln X + \ln Y$
\end{itemize}
\textbf{TMS de $X$ en fonction de $Y$} \pause
\begin{itemize}
\item Combien de $Y$ sacrifié pour davantage de $X$\pause
\item Formellement: augmente $X$ de $\Delta X$: quel est le changement $\Delta Y $ qui conserve indifférence?
\end{itemize}
\end{frame}

\begin{frame}{Calculer le TMS}
\textbf{TMS de $X$ en fonction de $Y$} \pause
\begin{itemize}
\item On doit avoir $\Delta Y$ tel que $U(X + \Delta X, Y + \Delta Y) = U(X,Y)$
\item Approximation de premier ordre: $$U(X+\Delta X, Y+ \Delta Y) \simeq  U(X,Y)+  \Delta X  \frac{\partial U}{\partial X} + \Delta Y \frac{\partial U}{\partial Y}$$\pause
$$ \Rightarrow \;\; TMS = \Delta Y/ \Delta X =  -\frac{\partial U}{\partial X}/ \frac{\partial U}{\partial Y}$$
\end{itemize}

\textbf{Exemple}
\begin{itemize}
\item $U(X,Y) = \ln X + \ln Y$
 $$TMS = \frac{\partial U}{\partial X}/ \frac{\partial U}{\partial Y} = -Y/X$$
\end{itemize}

\end{frame}

\begin{frame}{Par différentielle totale}

Prenons la différentielle totale:
\begin{align}
dU = \frac{\partial U}{\partial X}dX + \frac{\partial U}{\partial Y}dY
\end{align}
Posons $dU = 0$, alors 

\begin{align}
\frac{dY}{dX}\bigg\rvert_{dU=0} = -\frac{\partial U}{\partial X}/ \frac{\partial U}{\partial Y}
\end{align}

\end{frame}


\section{Contrainte budgétaire}

\begin{frame}{Contrainte Budgétaire}
\begin{itemize}
\item On ne peut pas dépenser plus que notre richesse $I$
\item Deux biens $X$, $Y$:  Contrainte: $p_X X + p_Y Y = I$ \\ Définie ce qui est abordable étant donné $I$\\
\item Résoudre pour $Y$ en terme de  $X$:     $Y = \frac{I - p_X X}{p_Y}$\\
Le taux de change entre $X$ and $Y$ en respectant la contrainte: $$\frac{dY}{dX} = -\frac{p_X}{p_Y}$$
\end{itemize}

\end{frame}

\begin{frame}\frametitle{Contraintes, la suite}
\textbf{Normalisation}
\begin{itemize}
\item  Contrainte budgétaire la même si prix et richesse multiplié par même constante $\lambda$.
\item On peut acheter les mêmes biens.\pause
\item Normalisons $p_Y = 1$. Alors $Y = I - p_X X$. $p_X$ est maintenant en terme de quantité de $Y$ (numéraire). 
\end{itemize}
\textbf{Exercice C}: Montrer qu'une contrainte budgétaire ne change pas si on multiplie prix et revenu par $\lambda>0$. 
\end{frame}

\section{Choix du consommateur}

\begin{frame}\frametitle{Choix du consommateur}

\begin{itemize}
\item La contrainte est une donnée fixe. Quel est le plus haut niveau d'utilité sur cette contrainte?

\item On ne peut pas aller sur une courbe d'indifférence plus élevée que la contrainte

\item Toutes courbes plus basses est sous-optimale.

\item La courbe d'indifférence qui touche la contrainte (souvent tangente) donne le meilleur niveau de bien-être possible
\end{itemize}

\end{frame}

\section{Conditions mathématiques}

\begin{frame}\frametitle{Approche Directe}

\textbf{Le problème est}
\begin{itemize}
\item Maximise $U(X,Y)$ étant donné par contrainte $p_X X+ p_YY = I$
\end{itemize}

\pause
\textbf{Étape 1: Substituer la contrainte}
\begin{itemize}
\item Si achète $X$ alors on consomme $Y(X) = \frac{I - p_X X}{p_Y}$
\item Utilité seulement fonction de $X$: $V(X) = U(X,Y(X))$
\end{itemize}

\pause

\textbf{Étape 2: Maximiser sans contrainte}
\begin{itemize}
\item Voir que la solution de coin n'est pas optimale (cas $X= 0$ et $Y=0$)
\item Prendre condition de premier ordre 
\end{itemize}
\end{frame}

\begin{frame}{Approche Directe}

\textbf{La CPO}
\begin{itemize}
\item $$\frac{dV}{dX} = 0 \iff \frac{dU}{dX} + \frac{dY}{dX}\frac{dU}{dY} = 0$$
 $$\iff \frac{dU}{dX}\Bigg/\frac{dU}{dY} = \frac{p_X}{p_Y}$$
\end{itemize}
TMS sur la courbe d'indifférence = Pente de la contrainte budgétaire
\end{frame}

\begin{frame}\frametitle{Solution}

\textbf{Exercice D}: Trouvez les demandes pour $u(x,y) = XY$ sous la contrainte $p_X X + p_Y Y \le I $. 
\end{frame}

\section{Approche générale}

\begin{frame}{Lagrangien I}
On peut poser le lagrangien

\begin{align*}
L(X,Y,\lambda) = U(X,Y) - \lambda (p_X X + p_Y Y - I)
\end{align*}

\pause

Si on maximise: $\max_{X,Y,\lambda} L(X,Y,\lambda)$, les CPO sont

\pause

\begin{align*}
U'_X(X,Y) - \lambda p_X = 0 \\
U'_Y(X,Y) - \lambda p_Y = 0 \\
p_X X + p_Y Y = I
\end{align*}
\end{frame}

\begin{frame}{Lagrangien II}
En prenant le ratio des deux premières CPO, on a:

\begin{align*}
\frac{U'_X(X,Y)}{U'_Y(X,Y)} = \frac{p_X}{p_Y} \\
p_X X + p_Y Y = I
\end{align*}

\textbf{Exercice E}: Trouvez les demandes pour $u(X,Y) = XY$ tel que précédement mais par le lagrangien. 
\newline

\end{frame}

\begin{frame}{Utilité Indirecte}

L'utilité indirecte $V(p_X,p_Y,I)$ est le niveau d'utilité maximal à atteindre avec les prix $(p_X,p_Y)$ et 
$$ V(p_X,p_Y,I) = \max_{X,Y} \{ u(X,Y) : p_X X + p_Y Y \le I\} $$

\textbf{Exercice F}: Montrez que $\frac{\partial V}{\partial I} = \lambda$ où $V$ l'utilité indirecte. 
\end{frame}

\begin{frame}{Identité de Roy}

Si l'utilité indirecte est donnée par $V(p_X,p_Y,I)$ alors on peut retrouver les demandes par l'identité de Roy: 

$$ X^*(p_X,p_Y,I) = \frac{\partial{V(p_X,p_Y,I)}/\partial{p_X}}{\partial{V(p_X,p_Y,I)}/\partial{I}}  $$

\textbf{Exercice F}: Montrez que ceci est vrai en utilisant le théorème de l'enveloppe. 
\end{frame}

\end{document}
