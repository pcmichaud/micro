\documentclass[11pt, oneside,french]{article}   	% use "amsart" instead of "article" for AMSLaTeX format
\usepackage[frenchb]{babel}
\usepackage{geometry}                		% See geometry.pdf to learn the layout options. There are lots.
\geometry{letterpaper}                   		% ... or a4paper or a5paper or ... 
\usepackage[T1]{fontenc}
\usepackage[utf8]{inputenc}
\usepackage{amsmath}
\usepackage{graphicx}				% Use pdf, png, jpg, or eps§ with pdflatex; use eps in DVI mode
								% TeX will automatically convert eps --> pdf in pdflatex		
\usepackage{amssymb}



\title{Exercices on Risk}
\author{20851-W2020}
\date{}							% Activate to display a given date or no date

\begin{document}
\maketitle

\begin{enumerate}
    \item A consumer has preferences given by $u(X) = \log X$. She currently has a lottery ticket she is willing to sell. There is a 75\% chance she wins 100 dollars and a 25\% chance she looses 100 dollars. Her current wealth is 500 dollars. What is the minimum price she is willing to sell the ticket for? \textit{Answer: She is willing to accept at least 42.13 dollars}. 
    \item A entrepreneur meets with the Dragons and one of them makes an offer for two-thirds of profits against an investment of 5,000\$ in his business. He himself would need to put in 5,000\% (half of his savings) and would get only one third of profits. He has a 25\% chance of making 100,000\$ and a 75\% chance of loosing his investment. He has preferences represented by $u(X) = \sqrt{X}$. Should he accept the proposal? If he refuses, what is the offer he could counter-offer with that would make him indifferent between starting the business or dropping his plan? \textit{Answer: He refuses the current offer and would prefer to not start the business. He is willing to counter offer anything less than 64.7\% of profits in exchange for a 5,000\$ investment.} 
    \item Show that those picking $L_1$ over $L_2$ and picking $L_4$ over $L_3$ in the Allais Paradox have preferences that are inconsistent with expected utility theory. \textit{Answer: Write down EU with u() unspecified. You get $L_1 \succ L_2$ if $u(50)>\frac{1}{11}u(0) + \frac{10}{11}u(250)$ but the opposite if $L_4 \succ L_3$. A contradiction.}
    \item Show that no matter the subjective fraction of while balls (among those white or black) in the Ellsberg Paradox, those picking $L_1$ over $L_2$ and $L_4$ over $L_3$ do not behave according to expected utility theory. \textit{Answer: Write down EU with p, the probability of white (as a fraction of black or white) unspecified. Then $L_1 \succ L_2$ iff $p<0.5$ but the opposite, $p>0.5$ if $L_4 \succ L_3$. A contradiction.}

\end{enumerate}

\end{document}  