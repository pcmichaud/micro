\documentclass[handout]{beamer}
\usepackage[frenchb]{babel}
\usepackage[T1]{fontenc}
\usepackage[utf8]{inputenc}
\usepackage{graphicx}

% functions to plot
\def\func(#1){(#1)*(1-(#1))}
\hypersetup{colorlinks = true,linkcolor = blue,urlcolor  = blue}
            
\newcommand{\qGraph}[1]{\begin{center} \includegraphics[width =
\textwidth]{#1}\end{center}}

\newcommand{\mcl}{\mathcal}


\newenvironment{iPar}[1]{\textbf{#1} \begin{itemize}}{\end{itemize}}

\newcommand{\inc}{{inc}}
\newcommand{\cp}{{cmp}}
\newcommand{\bull}{$\bullet\;$} 

\newcommand{\esp}{\mathbf{E}} \newcommand{\ul}[1]{\underline{#1}}
\newcommand{\ol}[1]{\overline{#1}} \newcommand{\ora}[1]{\textbf{#1}}
\newcommand{\wh}{\widehat}
\newcommand{\mdp}{\medskip \pause}
\newcommand{\mc}{\mathcal}

\title{Strategic Behavior of Firms}
\author{Microeconomics \\ 20851}
\date{}

\begin{document}

\frame{\titlepage}

\begin{frame}\frametitle{Itinerary}

\begin{iPar}{Until now}
\item Consumer choice
\item Risk and time
\item Measuring welfare
\item Exchange
\item Production
\end{iPar}\mdp

\begin{iPar}{This class: Strategic Behavior of Firms}
\item Monopoly
\item Cournot Oligopoly
\item Price Discrimination
\end{iPar}\mdp

\end{frame}

\section[Outline]{}
\frame{\tableofcontents}

\section{Monopoly}

\begin{frame}{Monopoly: Price Manipulation}

\begin{iPar}{First Welfare Theorem} \item The market is Pareto efficient.  \item Firms take the price as given: they do not internalize the impact of their choice of production on price (\textit{price taker}).
\item Is this a good assumption? \end{iPar} \mdp

\begin{iPar}{Perfect competition} \item A large number of firms \item
All small in size \item The production of each has a small effect on total production $\Rightarrow$ fixed price reasonable \item e.g.: market for meat, cereals, coffee... But oil, diamonds? \end{iPar}

\end{frame}


\begin{frame}{Behavior of the Firm}
\begin{iPar}{Setup}
\item  Good $X$ and the \textit{numéraire} $M$.\smallskip

 Utility is quasi-linear $U(X,M) = V(X) + M$,
 
  $V(X)$ concave, $V'(0) = + \infty$
\item  Endownment $I_0$ of $M$, $X$ produced by a single firm, price $p$ (price of X since price of $M$ normalized to one). 
\item   The firm produces $X$ with a cost function $C(X)$. Assume $C'(0) = 0$
\end{iPar} \mdp 
\textbf{Exercise A} Find the equilibrium price if the firm is a \textit{price taker} for $V(X) = \sqrt{X}$ and $C(X) = X^2$
\end{frame}


\begin{frame}\frametitle{Strategic Price Manipulation}

\begin{itemize} \item The firm correctly anticipates that increasing its supply may lower the price on the market. 
\item The firm picks $X$ so it maximizes $$\Pi = p(X)X - C(X)$$
\item FOC yields $$\frac{d\Pi}{dX} = 0 \iff p'(X)X + p(X) = C'(X)$$
\item Produces $X_{PM}$ such that  $   p(X_{PM}) + p'(X_{PM}) X_{PM} = C'(X_{PM}) $ \mdp
\item marginal price-effect + infra-marginal price-effect  = marginal cost. The infra marginal effect captures the impact of expanding production on the value from existing production.\end{itemize}
\end{frame}



\begin{frame}\frametitle{Strategic Price Manipulation (II)}

\textbf{Exercise B} Find the equilibrium price if the firm is strategic for $V(X) = \sqrt{X}$ and $C(X) = X^2$. Compare graphically to the case when the firm is price-taker. 

\textbf{Exercise C} : In the exercise above, find an expression for the equilibrium price as a function of the price elasticity of demand.  

\end{frame}

\begin{frame}\frametitle{Monopoly}

\begin{itemize} \item At optimal production $X_{PM}$ $$p(X_{PM}) + p'(X_{PM})X_{PM} = C'(X_{PM}) $$ 
Given, $p'(X_{PM})<0$, then $ p(X_{PM}) > C'(X_{PM})$. \item The monopoly fixes production at a price higher than marginal cost. \end{itemize}

\begin{itemize}\item If production of a firm has no impact on the price, it produces $X_{PT}$ s.t. $$p(X_{PT}) = C'(X_{PT})$$ \item When the firm internalizes its effect on the price (has market power), it produces $X_{PM}$ such that  $$p(X_{PM}) + p'(X_{PM})X_{PM} > C'(X_{PM})$$ \item This implies $X_{PM} < X_{PT}$.
\end{itemize}

\end{frame}

\begin{frame}\frametitle{Monopoly}

A (artificial) monopoly is inefficient: 

\begin{itemize}
    \item The monopoly makes profits. Those profits are given by: $$ \Pi_{PM} = (p(X_{PM}) - p(X_{PT}) X_{PM} $$. The price taker does not make profits. 
    \item Those profits reduce consumer surplus by the same amount. So no total welfare loss from this transfer. 
    \item The monopoly imposes a welfare loss because units $X_{PT}>X_{PM}$ have $V'(X)-C'(X)>0$. 
\end{itemize}
\textbf{Exercise D}: Show the profits and the rents in the graph done for Exercise B. 

\end{frame}

\section{Oligopoly}

\begin{frame}\frametitle{Oligopoly} \begin{itemize} \item Oligopolies are markets where a (small) number of firms produce a good. Duopolies are a special case with two firms. 
\item Duopoly: Two identical firms produce quantities $X_A$ and $X_B$ \item Representative consumer with $U(X,M) =V(X) + M$ and $$ V(X) = D_0 X - \frac{\alpha}{2} X^2$$   \item Costs given by $C_A(X) = C_B(X)= c \times X$\end{itemize} \mdp

\begin{iPar}{Inverse demand}
\item  $P(X) = V'(X) = D_0 - \alpha X$
\end{iPar}
\end{frame}


\begin{frame}{What should we compare with?}

\begin{iPar}{Counterfactual is efficient production (Pareto sense)}
\item Pick $X$ such that marginal cost is equal to marginal valuation
\begin{eqnarray*} &\iff& c = D_0 - \alpha X \\ &\iff& X_{Pareto} = \frac{D_0 - c}{\alpha} \end{eqnarray*}
\end{iPar} \mdp

 
\begin{iPar}{In comparison, the monopoly }
\item Maximizes $$\Pi(X) =  P(X) X - c X = (D_0- \alpha X) X - c X$$
\item Produces $$X_{Mon} = \frac{D_0 - c}{2\alpha} = \frac{1}{2} X_{Pareto}$$

\end{iPar}

\end{frame}


\begin{frame}{Cournot Competition -- Two firms}
\begin{iPar}{Best response curves (BR)}
\item Two identical firms
\item If firm $B$ produces $X_B$, what is the optimal production $X_A$ of firm $A$? \pause
\item Profit $$\Pi_A = P \times X_A - C(X_A) = [D_0 - \alpha(X_B + X_A)]X_A - C(X_A)$$
\item FOC is $$ D_0 -   \alpha(X_B + X_A) - \alpha X_A - c  = 0$$
\item Best response (BR) at $X_B$ is  $$X_A = \frac{D_0 -  c - \alpha X_B}{ 2 \alpha} = \frac{D_0 - c}{2\alpha} - \frac{1}{2}X_B$$
\end{iPar}
\end{frame}


\begin{frame}{Cournot Competition -- Duopoly}
\begin{itemize}
\item If firm $A$ produces $X_A$, best response from $B$ is to pick
$$X_B =   \frac{D_0 -  c - \alpha X_A}{ 2 \alpha} = \frac{D_0 - c}{2\alpha} - \frac{1}{2}X_A $$
\end{itemize}\mdp

\begin{iPar}{Recall from Intro to Microeconomics principles of Game theory}
\item \textbf{Nash Equilibrium:} No firm can benefit from a unilateral change on her part. 

\item If $X^*_A$ and $X^*_B$ are equilibrium quantities, we must have that $A$ best responds in an optimal way to $X^*_B$ and firm $B$ best responds in an optimal way to $X^*_A$
\end{iPar}
\end{frame}


\begin{frame}{Equilibrium}
We have the best response functions
\begin{itemize}
\item  $$X^*_A =  \frac{D_0 - c - \alpha X_B^*}{ 2 \alpha} \quad and \quad  X^*_B = \frac{D_0 - c -  \alpha X_A^*}{2 \alpha} $$

\item Since symmetric firms: $X^*_A = X^*_B = X^*$ gives $$X^*_A = X^*_B = X^* = \frac{D_0 - c}{3\alpha}$$
\item Total production  $$2 X^* = \frac{2}{3} \frac{D_0 - c}{\alpha}$$
\end{itemize}

\end{frame}

\begin{frame}{Some Exercises}

\textbf{Exercise E}: Plot the two best response functions in the space ($X_A,X_B$). Plot the equilibrium outcome. \\
\textbf{Exercise F}: Starting from $X_A = \frac{D_0 - c}{2\alpha}$, $D_0=10,c=1,\alpha=1$, show how applying the best response functions of each firm sequentially would eventually lead you to the equilibrium outcome. \\ 
\textbf{Exercise G}: Assume firm A has marginal cost $c_A$ and firm B, $c_B$, with $c_A=2>c_B=1$. Also assume $D_0=10,\alpha=1$. Find the equilibrium quantities of each firm.  

\end{frame}

\begin{frame}{Cournot Competition -- Increasing Competition}
\begin{iPar}{With more firms}
\item N identical firms $F_1,F_2,\cdots,F_N$
\item Given $X_2, X_3, \cdots, X_N$, firm $F_1$ produces $X_1$ to maximize $$\Pi_1 = [D_0 - \alpha(X_1 + X_2 + \cdots + X_N)]X_1 - c\times X_1$$
\item Best response is $$X_1 = \frac{D_0 - c - \alpha[X_2 + X_3 + \cdots + X_N]}{ 2 \alpha}$$
\end{iPar}

\textbf{Exercise H}: Show how one obtains the best response function in the example above. 

\end{frame}

\begin{frame}{Cournot competition}

\begin{iPar}{Equilibrium with many firms}
\item $X_1$ optimal given $X_2, X_3, \cdots, X_N$
\item $X_2$ optimal given $X_1,X_3,X_4, \cdots, X_N$
\item $\cdots$
\item $X_N$ optimal given $X_1,X_2, \cdots, X_{N-1}$
\item $N$ equations, exploiting symmetry yields $$X_1^* = X_2^* = \cdots = X_N^* = \frac{D_0 - c}{(N+1) \alpha}$$ \item Total production $$\frac{N}{N+1} \frac{D_0 -c}{\alpha}$$
\end{iPar}

\end{frame}

\begin{frame}{Competition}


\begin{iPar}{Firm with no market power}
\item A firm with no market power picks $X_0$  such that price equals marginal cost $ p(X) = c$
\end{iPar}\mdp

\begin{iPar}{Competition}
\item If one firm acts as monopoly, total production is $$X^{Tot}_1 = \frac{1}{2}\frac{D_0-c}{\alpha} \quad  and \quad \textrm{price } p_1 = \frac{1}{2} D_0  + \frac{1}{2} c$$
\item If two firms, total production $$X^{Tot}_2 = \frac{2}{3}\frac{D_0-c}{\alpha} \quad  et \quad \textrm{price } p_2  = \frac{1}{3}D_0 +\frac{2}{3}c$$
\item If $N$ firms , total production $$X^{Tot}_N = \frac{N}{N+1}\frac{D_0-c}{\alpha} \quad  and \quad  p_N = \frac{1}{N}D_0 +\frac{N}{N+1}c$$

\end{iPar}
\end{frame}


\begin{frame}{Interesting Result}
\begin{iPar}{With many firms, cournot competition converges to perfect competition (price taking behavior):}
\item When $N$ is large, $p(X_N) \to c$, $$ \lim_{N \to \infty} \left( \frac{1}{N}D_0 +\frac{N}{N+1}c \right) = 0 + c = c$$
\item In other words, price equals marginal cost.
\item Firms cannot affect the price when too many firms in the market. They become price taker and the rents from strategic behavior are eliminated. 
\end{iPar}
\end{frame}

\section{Bertrand Price Competition}

\begin{frame}{Competing on Price}
\begin{itemize}
   \item  In some markets, capacity is illimited. Think of retail. 
   \item Firms compete on the price they post (think of a flea market). 
   \item Competition among a few sellers is best modelled using Bertrand price competition
    \item Take $N$ identical firms with marginal cost $c_i=c \quad \forall i=1,...,N$. 
    \item First assume, $p_i = c \quad \forall i$. \\ \\
    \item If one firm $j$ deviates and posts $p_j>c$, she sells nothing (looses her market share). If she posts, $p_j<c$, she makes a loss. \item Equilibrium (nash) solution: $p_i=c \quad \forall i $. As opposed to Cournot, Bertrand yields a competitive market outcome (Pareto optimal)!
    \item Key assumption: Consumer can observe prices of all competitors without cost. Good homogeneous.
\end{itemize}    
\end{frame}

\begin{frame}{Bertrand Price Competition}
\begin{itemize}
    \item Cournot better suited for things that take time to build, capacity (e.g. airplanes, etc).  
   \item  In some markets, capacity is almost illimited and quantity adjusted very quickly. e.g. retail. 
   \item Firms compete on the price they post (think of a flea market with many kiosks close to each other selling the same thing). 
   \item Competition among a few sellers  is best modelled using Bertrand price competition

\end{itemize}    
\end{frame}

\begin{frame}{Bertrand Price Competition}
\begin{itemize}
    \item Take $N$ identical firms with marginal cost $c_i=c \quad \forall i=1,...,N$. 
    \item First assume, $p_i = c \quad \forall i$. \\ \\
    \item If one firm $j$ deviates and posts $p_j>c$, she sells nothing (looses her market share). If she posts, $p_j<c$, she makes a loss. \item Equilibrium (Nash) solution: $p_i=c \quad \forall i $. As opposed to Cournot, Bertrand yields a competitive market outcome (Pareto optimal)!
    \item Key assumption: Consumer can observe prices of all competitors without cost. Good homogeneous.
\end{itemize}

\end{frame}

\begin{frame}{Other Important Topics}
The field of industrial organization deals with strategic decisions of firms in different market settings. 

Other topics of interest are: 
\begin{itemize}
\item Product differentiation (monopolistic competition)
\item Collusion
\item Price discrimination
\end{itemize}

These are covered in courses like \textit{Comprendre la concurrence} (BAA) and \textit{Industrial Organization} (M.sc). 

\end{frame}
\end{document}




